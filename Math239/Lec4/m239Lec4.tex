\documentclass[12pt]{article}

\setlength\parindent{0pt}
\newcommand{\myt}[1]{\textbf{\underline{#1}}}

\usepackage{mathtools}
\usepackage{amssymb}

\title{\vspace{-15ex}Math 239 - Lecture 3\vspace{-1ex}}
\date{May 11th, 2015}
\author{Graham Cooper}

\begin{document}
	\maketitle
	\section*{Combinatoial Proofs}
	Recal ${n \choose k} = {n-1 \choose k} + {n-1 \choose k-1}$\\
	${7 \choose 4} = {6 \choose 3} + {6 \choose 4}$\\
	$= {6 \choose 3} + {5 \choose 3} + {5 \choose 4}$\\
	$= {6 \choose 3} + {5 \choose 3} + {4 \choose 3} + {3 \choose 3}$\\
	
	Identity: $ {n+k \choose n} = \sum_{i=0}^{k}{n+i - 1 \choose n-1}$\\
	
	Combinatorial proof: Let S be the set of all subsets of [n+k] of size n. So $|S| = {n+k \choose n}$.\\
	
	[IMAGE 1]\\
	
	For i = 0, k, let $S_i$ be all subsets of $[n+k]$ of size n whose largest element is n+i. Then each element of $S_i$ consists of n+k-i together with:\\
	
	[\{$\underset{n-1 spots}{\_,\_, ...}$ \underline{n+i}\} n spots] in [n +i - 1]\\
	So $|S_i| = {n+i-1 \choose n-1}$\\
	Since $S = S_0 \cup S_1 \cup ... \cup S_k$ is a disjoint union\\
	$\therefore |S| = \sum_{i=0}^{k}$ Identity holds.\\
	
	Hockey stick identity with pascal's triangle.\\
	
	\section*{Generating Series}
	Example: How many subsets of [3] have size k? Let S be all subsets of [3].\\
	
	Give each element $\delta$ of S a weight w where $w(\delta) = |\delta|$ (Related to the counting problem)\\
	
	Our problem becomes "How many elements of S have weight k?"\\
	\begin{tabular}{ c | c | c}
		$\delta in S$ & $W(\delta)$ & $x^{W(\delta)}$\\ \hline
		$\theta$ & 0 & 1 \\
		\{1\} & 1 & x \\
		\{2\} & 1 & x \\
		\{3\} & 1 & x \\
		\{1,2\} & 2 & $x^2$\\
		\{1,3\} & 2 & $x^2$\\
		\{2,3\} & 2 & $x^2$\\
		\{1,2,3\} & 3 & $x^3$\\
	\end{tabular}
	
	For each element $\delta$, contribute $x^{W(\delta)}$ to the "generating series" of S. Sum them all up. In this example, the generating series for S is $\Phi_s(x) = 1 + 3x + 3x^2 + x^3 = (1+x)^3$\\
	The coeff of $x^k$ records the answer to our counting problem.\\
	
	Definition: Given set S where each eleent $\delta \in S$ is given a non-negative integer weight $W(\delta)$, the generating series for S with respect to w is $\Phi_s(x) = \sum_{\delta \in S}x^{w(\delta)}$.\\
	
	Let $a_k$ be the number of elements in S of weight k.\\
	Then $\Phi_s(x) = \sum_{k \geq 0}a_kx^k$\\
	
	Example: How many subsets of [n] have size k?
	Let S be all subsets of [n].\\
	For any $\delta \in S$, define $w(\delta) = |\delta|$\\
	
	The number of elements in S of weight k is ${n \choose k}$:\\
	The generating series for S is $\Phi_s(x) = \sum_{k = 0}{n}{n \choose k}x^k = (1+x)^n$\\
	The answer is coeff of $x^k$ in $(1+x)^n$\\
	
	Example: How many ways can we throw 2 6-sided dice to get a sum of k?\\
	
	Let S = [6] $\times$ [6]\\
	For each $(a,b) \in S$, define w(a,b) = a + b.\\
	
	\begin{tabular}{c | c | c}
		(2,5) & wt 7 & $x^7$ \\ \hline
		(5,6) & wt 11 & $x^{11}$ \\ \hline
	\end{tabular}
	
	$\Phi_s(x) = \sum_{(a,b) \in S}x^{a+b}$ Coeff of $x^k$ is the number of ways to get a sum of k.\\
	
	$\Phi_s(x) = x^2 + 2x^3 + 3x^4 + 4x^5 + 5x^6 + 6x^7 + 5x^8 + 4x^9 + 3x^10 + 2x^11 + x^12$\\
	$ = (x + x^2 + x^3 + x^4 + x^5 + x^6)^2$\\
	
	
	
	
	
\end{document}
