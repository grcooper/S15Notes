\documentclass[12pt]{article}

\setlength\parindent{0pt}
\newcommand{\myt}[1]{\textbf{\underline{#1}}}

\usepackage{mathtools}
\usepackage{amssymb}

\title{\vspace{-15ex}Math 239 Lecture 1: Introduction\vspace{-1ex}}
\date{May 4th, 2015}
\author{Graham Cooper}

\begin{document}
	\maketitle
	Prof: 	Martin Pei\\
	Email:	mpei@uwaterloo.ca\\
			martin31315926@gmail.com\\
	Office Hours:\\
	MTF:	11:30am-12:20pm\\
	T:		3:30-5:30pm\\
	
	\section*{Part 1 Enumeration}
	
	We will convert problems into sets\\
	
	[n] = \{1, 2, 3, ... n\}\\
	
	$\mathbb{N} = \{1, 2, 3 ...\}$ (no zero)\\
	
	\subsection*{Cartesian Product:}
	If A, B are sets, then the cartesian product of A and B is:\\
	
	$A \times B = \{(a, b) | a \in A, b \in B\}$\\
	
	\myt{Example:} A = \{1, 2\}  B = \{2, 4, 6\}\\
	
	$A \times B = \{(1, 2), (1, 4), (1, 6), (2, 2), (2, 4), (2, 6)\}$\\
	- Order inside the pairs matter\\
	- Order of the pairs in the set does not matter\\
	
	If A and B are finite sets, then $|A \times B| = |A| \cdot |B|$\\
	
	$(A \times B) \times C \neq A \times(B \times C)$\\
	$((a, b), c) \neq (a, (b, c))$\\
	
	$|A \times B \times C| = |A| \cdot |B| \cdot |C|$\\
	
	$A^n = \{(a_1, a_2, ... a_n) | a_i \in A \} |A^n| = |A|^n$\\
	
	\myt{Example:} The results of throwing 2 different 6-sided dice can be enumerated by $[6] \times [6] or [6]^2$\\
	$ = \{(a, b) | a, b \in [6]\}$\\
	
	\subsection*{Disjoint Unions:}
	Let $S = S_1 \cup S_2$ and $S_1 \cap S_2 = \phi$ Then $|S| = |S_1| + |S_2|$\\
	
	$S = S_1 \cup ... \cup S_k$ and $S_i \cap S_j = \phi$ where $i \neg j$\\
	
	\myt{Example:} Let E be elements of $[6] \times [6]$ whose sum is even.\\
	
	Partition E into $E_1$ and $E_2$ where\\
	
	$E_1 = \{(a, b) \in [6] \times [6]$ \vline \ a, b are even \}\\
	$E_2 = \{(a, b) \in [6] \times [6]$ \vline \ a, b are odd \}\\
	
	$E_1 = {2, 4, 6} \times {2, 4, 6}$ $|E_1| = 3 \cdot 3 = 9$\\
	$E_2 = {1, 3, 5} \times {1, 3, 5}$ $|E_2| = 3 \cdot 3 = 9$\\
	
	So $|E| = |E_1| + |E_2| = 18$ $(E_1 \cap E_2) = \phi$
	
	\subsection*{Review Basic Counting}
	\subsubsection*{Permutations}
	How many ways can we arrange elements of [n] in a line?\\
	
	There are n choices for the first spot, n-1 for the second... down to 1 choice for the last spot so there are n! different ways.\\
	
	\subsubsection*{Combinations}
	How many subsets of [n] have size k?\\
	
	First pick k elements in order. There are n choices, then n-1 choices then ... then n - k + 1 choices.\\
	
	$ = \frac{n!}{(n - k)!}$\\
	
	Each subset of size k is counted k! times in order. So the number of subsets is: \\
	
	$\frac{n!}{k! \cdot (n - k)!} = {n \choose k}$ "N choose k"\\
	
	\subsubsection*{Binomial Theorem}
	$(1 + x)^n = \sum_{k = 0}^{n} {n \choose k} \cdot x^k$\\
	
	We will touch on this later in the course
	
	
\end{document}
