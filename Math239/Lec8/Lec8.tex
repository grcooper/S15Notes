\documentclass[12pt]{article}

\setlength\parindent{0pt}
\newcommand{\myt}[1]{\textbf{\underline{#1}}}

\usepackage{mathtools}
\usepackage{amssymb}

\title{\vspace{-15ex}Math 239 Lecture 8\vspace{-1ex}}
\date{May 22th, 2015}
\author{Graham Cooper}

\begin{document}
	\maketitle
	
	\section*{Product Lemma}
	Recall:\\
	
	Sets A,B with weight $\alpha$, $\beta$\\
	Set A $\times$ B, with weight w(a,b) = $\alpha(a) + \beta(b)$\\
	Then $\Phi_{A\times B}(x) \cdot \Phi_B(x)$\\
	
	\subsection*{Proof of the Product Lemma}
	$$\Phi_A(x) \cdot \Phi_B(x) = (\sum_{a \in A}x^{\alpha(a)})(\sum_{b \in B}x^{\beta(b)})$$
	$$= \sum_{a \in A}\sum_{b \in B}x^{\alpha(a)}x^{\beta(b)}$$
	$$\sum_{(a,b) \in A \times B}x^{\alpha(a) + \beta(b)}$$
	$$= \sum_{(a,b) \in A \times B}x^{w(a,b)}$$
	$$= \Phi_{A\times B}(x)$$
	
	\myt{Example:} Let $N_0 = \{0,1,2,3,...\}$ w(a) = a. Then:\\
	
	$$\Phi_{N_0}(x) = 1 + x + x^2 + x^3 + ... = \frac{1}{1-x}$$
	$\frac{1}{(1-x)^k}$ is the generating series for $N_0 \times N_0 ... \times N_0 = N_0^k$\\
	Where $w(a_1,a_2...a_k) = a_1 + a_2 + ... + a_k$ by product lemma.\\
	
	So $[x^n]\frac{1}{(1-x)^k}$ is the number of k tuples ($a_1...a_k) \in N_0^k$ where they sum to n.\\
	$\iff$ the number of non-negative integer solutions to $a_1 + a_2 + ,,, + a_k = n$\\
	
	In general, any solution $(a_1, ... a_k)$ corresponds to an arrangement of n 0's and k-1 1's\\
	$0^{a_1} | 0^{a_2} | ... | 0^{a_k}$\\
	So there are ${n+k-1 \choose k-1}$ of them. So $[x^n]\frac{1}{(1-x)^k} = {n+k-1 \choose k-1}$\\
	
	myt{Example:} How many ways can n identical pieces of sushi be eaten so Al eats at most 5, Bob eats at least 3 and Cam eats an even number?\\
	
	Model the problem as $(a,b,c) \in A \times B \times C$ where:\\
	$$A = \{0,1,2,3,4,5\}$$
	$$B = \{3,4,5,6 ...\}$$
	$$C = \{0,2,4,6 ...\}$$
	
	Define w(a,b,c) = a + b + c. Using $\alpha(a) = a$ for all A,B,C we can apply the product lemma.\\
	Then 
	$$\Phi_A(x) = 1 + x + x^2 + x^3 + x^4 + x^5 = \frac{1-x^6}{1-x}$$
	$$\Phi_B(x) = x^3 + x^4 + x^5 + x^6 ... = \frac{x^3}{1-x}$$
	$$\Phi_C(x) = 1 + x^2 + x^6 + ... = \frac{1}{1-x^2}$$
	
	So
	$$\Phi_{A\times B \times C}(x) = \Phi_A(x) \Phi_B(x) \Phi_C(x) = \frac{x^3(1-x^6)}{(1-x)^2(1-x^2)}$$
	The number of ways is $[x^n]\frac{x^3(1-x^6)}{(1-x)^2(1-x^2)}$\\
	
	\section*{Integer Compositions}
	\subsection{Definition}
	A k-tuple $(a_1, ... a_k)$ of positive integers is a \underline{composition} of n if $n=a_1 + ... a_k$ Such a composition is said to have k parts.\\
	
	\myt{example} Compositions of 5 include (1,3,1), (2,3), (1,1,1,2), (2,1,1,1), (5)\\
	
	\subsection*{notes}
	\begin{enumerate}
		\item Each part is at least 1
		\item Order of the parts matter
		\item The number 0 has 1 composition which has 0 parts ()
	\end{enumerate}
	
	\myt{example:} How many compositions of n have k parts?\\
	$[n=4, k=3 : (1,1,2), (1,2,1), (2,1,1)]$\\
	
	The set of all compositions of k parts (disregardig n) is $N \times N \times N ... N = N^k$, one composition has the form $(a_1,...a_k)$\\
	
	Define $w(a_1, ... a_k) = a_1 + ... a_k$\\
	Use w(a) = a for each N. \\
	$$\Phi_N(x) = x^1 + x^2 + x^3 + ... = \frac{x}{1-x}$$
	By Product Lemma
	$$\Phi_{N^k}(x) = (\Phi_N(x))^k = \frac{x^k}{(1-x)^k}$$
	Our answer is:
	$$[x^n]\frac{x^n}{(1-x)^n}$$
	
	
	
	
\end{document}
