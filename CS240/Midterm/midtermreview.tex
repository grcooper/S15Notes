\documentclass[12pt]{article}

\setlength\parindent{0pt}
\newcommand{\myt}[1]{\textbf{\underline{#1}}}

\usepackage{mathtools}
\usepackage{amssymb}

\title{\vspace{-15ex}CS240 Midterm Review\vspace{-1ex}}
\date{June 20th, 2015}
\author{Graham Cooper}

\begin{document}
	\maketitle
	
	Topics:\\
	\begin{itemize}
		\item First principals
	\end{itemize}
	
	\section*{First Principals}
	\subsection*{1)}
	$$10n^2 + 11n + 12 \in O(n^2)$$
	When n $\geq$ 1\\
	$$10n^2 + 11n + 12 \leq 10n^2 + 11n^2 + 12n^2$$
	$$= 33n^2$$
	$n_0 = 1, c = 33$\\
	
	\subsection*{2)}
	$$n+ (logn)^2 \in O(nlogn)$$
	when $n \geq 2$\\
	$$n + (logn)^2 \leq n + n$$
	$$\leq 2nlogn$$
	$c = 2, n_0 = 2$
	
	\subsection*{3)}
	$$10n^2 - 11n - 12 \in \Omega(n^2)$$
	When $n^2 \geq 11n = n \geq 11$\\
	$$10n^2 - 11n - 12 \geq 10n^2 - n^2 - 12$$
	$$= 9^2 - 12$$
	When $n^2 \geq 12 = n^2 \geq \sqrt{12}$
	$$\geq 9n^2 - n^2$$
	$$= 8n^2$$
	
	c = 8, $n_0 = 11$\\
	
	\subsection*{4)}
	$$5n + 15logn - 10\sqrt{n} \in \Omega(logn)$$
	when $n \geq 2$\\
	$$5n + 15logn - 10\sqrt{n} \geq 5n - 10\sqrt{n}$$
	when $10\sqrt{n} \leq n = 10 \leq \sqrt{n} = 1oo \leq n$\\
	$$\geq 5n - n$$
	$$= 4n$$
	
	\subsection*{5)}
	$$3n^2 10n + 15logn + 2 \in \Theta(n^2)$$
	$$3n^2 - 10n + 15logn + 2 \leq 3n^2 + 15logn + 2$$
	if n $\geq 1$\\
	$$\leq 3n^2 + 15n^2 + 2$$
	$$\leq 3n^2 + 15n^2 + 2n^2$$
	$$= 20n^2$$
	$c_2 = 20$\\
	
	$$3n^2 - 10n + 15logn + 2 \geq 3n^2 - 10n$$
	when $10n \leq n^2 = 10 \leq n$\\
	$$\geq 3n^2 - n^2$$
	$$= 2n^2$$
	
	$$2n^2 \leq 3n^2 - 10n + 15logn + 2 \leq 20n^2$$
	when $n \geq 10$
	$c_1 = 2, c_2 = 20, n_0 = 10$\\
	
	\subsection*{6)}
	$n \in o(n^2)$\\
	$$n \leq cn^2$$
	$$1 \leq cn$$
	$$\frac{1}{2} \leq n$$
	
	$n_0 = \frac{1}{c}$\\
	$c = 0.1$\\
	
	\subsection*{7)}
	$$cos(n) \in o(n)$$
	$$cos(n) \leq cn$$
	$$cos(n) \leq 1 \leq cn$$
	$$\frac{1}{c} \leq n$$
	
	$n_0 = \frac{1}{c}$\\
	
	\subsection*{8)}
	
	$n^n \in w (n^20)$\\
	
	$$n^n \geq cn^20$$
	$$\frac{n^n}{n^20} \geq c$$
	$$n^{n-20} \geq c$$
	the below occurs when n $\geq$ 21\\
	$$n^{n-20} \geq n \geq c$$
	The inequality holds when $n \geq max(c,21)$\\
	
	\section*{Loop Analysis}
	\subsection*{1)}
	\begin{verbatim}
	foo(n,m,k)
	1. for (i = 1 to n)
	2. -- for(j = 1 to i)
	3. -- -- for(l = 1 to m)
	4. -- -- -- print("hello")
	5. for(i = 1 to k)
	6. -- for(j = 1 to 600)
	7. -- -- print("WORld")
	\end{verbatim}
	
	line 4: $\Theta(1)$\\
	line 7: $\Theta(1)$\\
	lines 6-7 $\sum_{i=1}^{600}1 = 600 \in \Theta(1)$\\
	lines 5-7 $\sum_{n=1}^{k}\Theta(1) = \Theta(k)$\\
	lines 3-4 $\sum_{l = 1}^m 1 = m$\\
	lines 2-4 $\sum_{j = 1}^{l = 1}m  = i \cdot m$\\
	lines 1-4 $\sum_{i=1}^{n} i \cdot m$\\
	$= m \sum_{i = 1}^{n}i = \frac{m(n(n+1))}{2}$\\
	
	Total = $\frac{m(n(n+1))}{2} + 600k$\\
	$\in \Theta(mn^2 + k)$\\
	
	\subsection*{2)}
	\begin{verbatim}
	foo2(n,m)
	1. while (n > m)
	2. -- n = n/2
	3. -- for (i = 1 to m)
	4. -- -- print("x")
	\end{verbatim}
	
	After t iterations\\
	m' = m\\
	n' = $\frac{n}{2^t}$\\
	$\frac{n}{2^2} \leq m$\\
	
	$t \geq log(\frac{n}{m}) = log(n) - log(m)$\\
	
	Aside:\\
	$\frac{n}{2^{t-1}} \geq m \geq \frac{n}{2^t}$\\
	$\frac{2n}{m} > 2^t \geq \frac{n}{m}$\\
	$log(\frac{2n}{m}) > t \geq log(\frac{n}{m})$\\
	$= 1+ log(\frac{n}{m})$\\
	
	$$\sum_{j=1}^{log(\frac{n}{m})}\sum_{i=1}^{m}1$$
	$$=\sum_{j=1}^{log(\frac{n}{m})}m$$
	$$ = mlog(\frac{n}{m})$$
	
	\subsection*{3)}
	\begin{verbatim}
	foo3(n)
	1. while(n > 1)
	2. -- foo3(n-1)
	3. -- n--
	4. for(i = 1 to n)
	5. -- print("y")
	\end{verbatim}
	
	After t iterations\\
	$$n' = n-t$$
	$$n-2 \leq 1 < n-t + 1$$
	$$t = n$$
	
	$$T(n) = \sum_{i = 1}^{n}(T(i) + 1) + \sum_{i = 1}^{n}1$$
	$$ = T(1) + T(2) + ... + T(n-1) + n - 1 + n$$
	$$T(n-1) = T(1) + T(2) + ... + T(n-2) + n-2 + n-1$$
	$$T(n) - T(n-1) = T(n-1) + (n-1+n) - (n - 2 + n - 1)$$
	
	$$T(n) = 2T(n-1) + \Theta(1)$$
	$$T(n) = 2(2(T(n-2) + \Theta(1)) + \Theta(1)$$
	$$2^3T(n-3) + 6\Theta(1)$$
	$$2^kT(n-k) + \Theta(1)$$
	$$= 2^kT(1) + \sum_{i=0}^{k-1}2^i$$
	k = n-1 in this case\\
	$$= 2^{n-1} \cdot 1 + \sum_{i=0}^{n-2}2^i$$
	$$= \Theta(2^n)$$
	
	\subsection*{4)}
	\begin{verbatim}
	foo4(n)
	1. if (n = 1) return
	2. for (i = 1 to 5)
	3. -- foo4(n/3)
	4. print("HI")
	\end{verbatim}
	
	$$T(n) = 5T(\frac{n}{3}) + 1$$
	$$T(\frac{n}{3}) = 5(5T(\frac{n}{9}) + 1) + 1$$
	$$T(n) = 5(5(5T(\frac{n}{27}) + 1) + 1) + 1$$
	$$5^{log_3n} + 1 + 5 + 25 + 5^{log_3n}$$
	$$T(n) = S^{log_3n} + 1 + \sum_{i = 1}^{log_2n}5^i$$
	$$\in \Theta(n^{log_35})$$
	
	\section*{True false}
	\subsection*{1)}
	$$f(n) \in \Theta(g(n)) \implies f(n) \in o(g(n)) OR f(n) \in w(g(n))$$
	
	counter example: \\
	f(n) = max(1,nsin(n))\\
	g(n) = max(1, ncos(n))\\
	
	\subsection*{2)}
	The average runtime of $f(n) \in \Theta(n)$,\\
	- is the worst case runtime O(n)\\
	- is the worst case runtime $\Omega(n)$\\
	- is the best case O(n)\\
	
	1 is false, 2 is true, 3 is true\\
	
	
	
	
\end{document}
