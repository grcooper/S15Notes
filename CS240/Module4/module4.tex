\documentclass[12pt]{article}

\setlength\parindent{0pt}
\newcommand{\myt}[1]{\textbf{\underline{#1}}}

\usepackage{mathtools}
\usepackage{amssymb}
\usepackage{tikz,ifthen,amsmath,amssymb,fancyhdr,comment,lastpage}

\title{\vspace{-15ex}Module 4 - Dictionaries and Balanced Search Trees\vspace{-1ex}}
\date{June 2nd, 2015}
\author{Graham Cooper}

\begin{document}
	\maketitle
	\section*{Dictionaries}
	\begin{itemize}
		\item An ADT
		\item Data (key, value) pairs
		\item operations: search, insert, delete
	\end{itemize}
	
	Data Structures for Dictionaries:
	\begin{itemize}
		\item unsorted array or linked list
		\begin{itemize}
			\item search: O(n)
			\item insert: O(1)
			\item Delete: O(n)
		\end{itemize}
		\item sorted array
		\begin{itemize}
			\item search - binary search O(logn)
			\item insert O(n)
			\item delete O(n)
		\end{itemize}
		
	\end{itemize}
	
	\subsection{BST}
	
	\begin{center}\begin{tikzpicture}[
		level distance=45 pt,
		every node/.style={circle,draw},
		level 1/.style={sibling distance=200 pt},
		level 2/.style={sibling distance=100 pt},
		level 3/.style={sibling distance=60 pt}
		]
		\node {20}
		child {node {10}
			child {node {5}
				child {node {1}}
				child {node {8}}
			}
			child {node {15}
				child [missing]
				child {node {16}}
			}
		}
		child {node {50}
			child [missing]
			child {node{70}
				child {node {60}}
				child {node {80}}
			}	
		}
		;
		\end{tikzpicture}\end{center}
	
	Insert 6\\
	
	\begin{center}\begin{tikzpicture}[
		level distance=45 pt,
		every node/.style={circle,draw},
		level 1/.style={sibling distance=200 pt},
		level 2/.style={sibling distance=100 pt},
		level 3/.style={sibling distance=60 pt}
		]
		\node {20}
		child {node {10}
			child {node {5}
				child {node {1}}
				child {node {8}
					child {node{6}}
					child [missing]	
				}
			}
			child {node {15}
				child [missing]
				child {node {16}}
			}
		}
		child {node {50}
			child [missing]
			child {node{70}
				child {node {60}}
				child {node {80}}
			}	
		}
		;
		\end{tikzpicture}\end{center}
	
	Delete in a BST\\
	\begin{itemize}
		\item if n is a leaf - just delete it
		\item if n is a node with one child, replace it with its child
		\item if n has two children, replace with the predecessor (rightmost on the left) or sucessor (left most)
	\end{itemize}
	
\end{document}
