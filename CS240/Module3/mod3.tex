\documentclass[12pt]{article}

\setlength\parindent{0pt}
\newcommand{\myt}[1]{\textbf{\underline{#1}}}

\usepackage{mathtools}
\usepackage{amssymb}

\title{\vspace{-15ex}CS 240 Module 3\vspace{-1ex}}
\date{May 4th, 2015}
\author{Graham Cooper}

\begin{document}
	\maketitle
	
	\section*{Selection}
	Given an array A$[a...n-1]$ and $0 \leq k \leq n-1$ return the kth largest element in A.\\
	
	\subsection{1) Selection-sort Idea:}
	Scan A k times, deleting max each time.\\
	Cost: $\Theta(kn)$\\
	\subsection{2)}
	Sort A, return A[n-k]\\
	Cost: $\Theta(nlogn)$\\
	
	\subsection{3)}
	Scan the array once, and keep k largest seen so far in the min-heap.\\
	Cost: $\Theta(nlogk)$\\
	Eg: [6,5,3,8,7,4], k =3\\
	We put in 6, 5 then 3 into the min heap. After we look at the rest of the elements and keep the min heap the size of k and add new elements if an element in the array is larger than the root of the min-heap. Continue through the array and at the end pick the root of the min heap.
	
	\subsection{4)}
	Heapify(A) then call deleteMax k times.\\
	Cost: $\Theta(n + klogn)$ For median selection (k = n/2) then it is the same as sorting so $\Theta(n)$\\
	
	
	
	\section*{Partition Algorithm}
	Given an array A[0...n-1] and $0 \leq k \leq n-1$, find the element at position k of the sorted A.\\
	
	\myt{Observation:}\\
	A = $[\overset{0}{7}, \overset{1}{3}, \overset{2}{2}, \overset{3}{4}, \overset{4}{6}, \overset{5}{1}]$\\
	Sorted(A) = [1, 2, 3, 4, 6, 7]\\
	What is the position of A[3](4) in the sorted A. the answer is the number of elements $<$ A[3] in A[0..2] and A[4,5]\\
	
	\myt{Idea:} choose one element (pivot) and partition the data into: (items $<$ pivot), pivot, (items $>$ pivot). If position(pivot) == k, done, otherwise, continue either on the left or on the right, depending on the position of the pivot.\\
	
	WHAT WE WANT TO DO:\\
	
	Implicit A = $[\overset{0}{9}, \overset{1}{4}, \overset{2}{5}, \overset{3}{8}, \overset{4}{6}, \overset{5}{3}, \overset{6}{2}]$\\
	
	Lets pick A[2] as the pivot, swap A[2] and A[0]\\
	A = [5,4,9,8,6,3,2]\\
	
	\myt{Idea:} Find the outermost wrongly positioned pair and swap.\\
	
	advance i, backup j.\\
	A = [5, 4, 9, 8, 6, 3, 2]\\
	$i < j$ so we should swap i\\
	A = [5, 4, 2, 8, 6, 3, 9]\\
	Advance i, backup j\\
	A = [5, 4, 2, 8 ,6 ,3, 9]\\
	$i < j$ swap i\\
	A = [5, 4, 2, 3, 6 ,8 ,9]\\
	advance i, backup j\\
	A = [5, 4, 2, 3, 6, 8 ,9]\\
	$j < i$ stop, swap, A[0] wiht A[j]\\
	A = [3, 4, 2, 5, 6, 8, 9]\\
	Return 3.\\
	
	\subsection*{Quick Select(A,K)}
	P = choosePivot(A)\\
	i = partition(P)\\
	if i = k\\
	return A[i]\\
	if i $>$ k:\\
	return QuickSelect(A[0...i-1], k)\\
	if i $<$ k:\\
	return QuickSelect(A[i+1...n-1], k-i-1)\\
	
	\subsubsection{Cost of Quick Select}
	Let T(n) e cost of QuickSelect\\
	T(n) = $\Theta(n) + $\\
	$\Theta(1)$, if n = k\\
	T(i) if $i > k$\\
	T(n-i-1), if $i < k$\\
	
	\myt{Best Case:} T(n) = $\Theta(n)$ if i = k\\
	(first chosen pivot if the element at position k, no recursive calls)\\
	
	\myt{Worst Case:} i = 0 or i = n-1\\
	Recursive call has size n - 1\\
	(if we pick the first element as the pivot, then an array sorted in ascending or descending order will give the worst case runtime.)\\
	
	T(n) = \\
	d if n = 1\\
	T(n-1) + cn if n $\geq$ 2\\
	T(n) = cn + c(n-1) + c(n-2) + ... + c(2) + d\\
	= $c)\frac{n(n+1)}{2} - c + d \in \Theta(n^2)$\\
	
	\myt{What if hte partition is balanced}\\
	A[p] is always close to median\\
	T(n) = \\
	T($\frac{n}{2}) + cn$ if n $\geq$ 2\\
	d if n = 1\\
	
	Assume n is a power of 2: $2^x$\\
	$T(2^x) = c \cdot 2^x + c \cdot 2^{x-1} + ... + c \cdot 2 + d$\\
	$ = c(2^{x+1} - 2) + d$\\
	$ = 2c(n-1) + d \in \Theta(n)$\\
	
	\myt{Average-Case analysis}: Average cost over all inputs of size n as function of n.\\
	\underline{Observation}: behaviour of QuickSelect depends on relative ordering, and not on actual values. [1,3,5,7] will yeild the same worst case behaviour as [4,5,6,7].\\
	
	Assume all keys are unique, $x_1, x_2,...x_n$ then there are n! possible orderings on these keys. and each ordering is equally likely\\
	
	After we pick the pivot, what iwll the split look like?\\
	\begin{tabular}{c | c}
		L(num of items) & R(num of items) \\ \hline
		0 & n-1 \\
		1 & n-2 \\
		... & ...\\
		k - 1 & n - k\\
		k & n - k - 1 \\
		k + 1 & n - k - 2\\
		... & ... \\
		n - 1 & 0 \\
	\end{tabular} 
	
	For each choce of pivot (n possible pivots) there are (n-1)! permutations of non-pivot elements, each of the splits is equally likely\\
	
	After Partiction:\\
	A = [0...x...]\\
	Define T(n,k) an average cost for selecting kth item from a size n array.\\
	
	$$T(n,k) = cn + \frac{1}{n}T(n-1, k-1)x + \frac{1}{n}(n-2, k-2) + ...$$
	Put in summation notation.\\
	
	
\end{document}
