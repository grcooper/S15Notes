\documentclass[12pt]{article}

\setlength\parindent{0pt}
\newcommand{\myt}[1]{\textbf{\underline{#1}}}

\usepackage{mathtools}
\usepackage{amssymb}

\title{\vspace{-15ex}CS 241 Lecture 20\vspace{-1ex}}
\date{July 15th, 2015}
\author{Graham Cooper}

\begin{document}
	\maketitle
	
	\section*{Evaluating Expressions}
	\subsection*{tests}
	\begin{verbatim}
	test -> expr1 < expr2
	
	code(test) = 
	code(expr1) ($3 <- expr1)
	add$6, $3, $0
	code(expr2) ($3 <- expr2)
	slt $3, $6, $3
	\end{verbatim}
	Treat test $\rightarrow$ expr1 $>$ expr2 the same
	
	\begin{verbatim}
	test -> expr1 != expr2
	code(test) = 
	code(expr1) ($3 <- expr1)
	add $6, $3, $0 ($6,<- expr1)
	code(expr2) ($3 <- expr2)
	slt $5, $3, $6 ($5 = $3 < $6)
	slt $6, $6, $3 ($6 = $6 < $3)
	add $3, $5, $6 ($3 = $6 OR $7)
	\end{verbatim}	
	
	\begin{verbatim}
	test -> expr1 == expr2
	
	treat as NOT (expr1 != expr2)
	- as above and then (recall $11 = 1)
	sub $3, $11, $3 ($3 <- 1 - $3)
	\end{verbatim}
	Leaving $<=$ and $>=$ as excercises
	
	\subsection*{IF}
	\begin{verbatim}
	stmt -> if test { stmts1 } else {stmts2}
	code(stmt) =
	code(test) ($3 <- test)
	beq $3, $0, else
	code(stmts1)
	else: beq $0, $0, endif
		code(stmts2)
	endif:
	\end{verbatim}
	
	WE need to generate unique labels
	\begin{itemize}
		\item keep a counter x
		\item use x, endifx, truex for label names
		\item incrememnt x for each new if-statement
	\end{itemize}
	
	\subsection*{While:}
	\begin{verbatim}
	stmt -> while(test){stmts}
	-use a counter Y
	code(stmt) =
	loopY:
	code(test) ($3 <- test)
	code(stmts)
	beq $0, $0, loopY
	doneY: 
	\end{verbatim}

	Done assn 9\\
	
	\myt{Advice:} Generate comments along with mips instructions
	
	\section*{Pointers:}
	Need to do seven?? things\\
	\begin{enumerate}
		\item NULL
		\item Dereferencing
		\item address-of
		\item Pointer Comparison
		\item Pointer Arithmetic
		\item alloc/dealloc
		\item assignment through pointers
	\end{enumerate}	
	
	\subsection*{NULL}
	\begin{itemize}
		\item could use 0
		\item Use 1 (not a multiple of 4)
		\item If 1 is dereferenced, the machine will crash
	\end{itemize}
	
	\begin{verbatim}
	factor -> NULL
	code(factor) = add $3, $11, $0 ($3 <- 1)
	\end{verbatim}
	
	\subsection*{Dereference}
	\begin{verbatim}
	factor -> *expr
	code(factor = 
	code(expr) ($3 <- expr)
	lw $3, 0($3)
	\end{verbatim}
	
	\subsection*{comparisons}
	\begin{itemize}
		\item same as UNSIGNED int comparisons
		\item (there are no negative pointers)
		\item use sltu instead of slt
	\end{itemize}
	\subsubsection*{Type Checking 1}
	So when generating code for:\\
	test $\rightarrow$ expr1 $<$ expr2 etc...\\
	use slt if expr1, expr2 : int\\
	use sltu if expr1, expr2 : int*\\
	- rerun "typeof" (or just on one of them since the types will be the same)\\
	
	\subsubsection*{TypeChecking Better}
	\begin{itemize}
		\item add a "type" field to each tree node
		\item make "typed" procedure record each node's type (if it has one) in the node itself
		\item then the type will be available if needed
	\end{itemize}
	
	\subsection*{Pointer Arithmetic}
	\begin{verbatim}
	expr1 -> expr2 + term
	expr1 -> expr2 - term
	\end{verbatim}
	
	\begin{itemize}
		\item the exact meaning depends on the types involved
	\end{itemize}
	
	\subsubsection*{Addition}
	\begin{verbatim}
	expr1 -> expr2 + term
	if expr2, term : int as before (just add)
	if expr2 : int*, term : int
	- means expr2 + (4 * term)
	code(Expr1) =
	code(expr2)
	push($3)
	code(term) ($3 <- term)
	pop($5)
	mult $3, $4
	mflo $3
	add $3, $5, $3
	
	if expr2: int term: int* means (4 * expr2 + term)
	\end{verbatim}
	
	\subsubsection*{Subtraction}
	\begin{verbatim}
	expr1 -> expr2 - term
	if expr2 : int, term : int - as before
	if expr2 : int*, term : int - means expr2 - 4 * term
	- similar to int* + int case
	
	if expr2 : int*, term : int*
	- means (expr2 - term)/4
	
	code(expr1) =
	code(expr2)
	push($3)
	code(term)
	pop($5)
	sub $3, $5, $3
	div $3, $4
	mflo $3
	\end{verbatim}
	
	\subsection*{Assignment}
	\subsubsection*{Assnment through pointer dereference}
	LHS = address at which to store the value
	RHS = value
	
	\begin{verbatim}
	stmt -> ID = expr - already done
	*factor = expr
	code(stmt) = code(factor) ($3 <- factor = the address)
	push($3)
	code(expr) ($3 <- expr)
	pop($5) ($5 <- factor)
	sw $3, 0($5)
	\end{verbatim}
	
	\subsection*{Address-of}
	\&lvalue
	lvalue is ID, or STAR factor
	\begin{verbatim}
	if expr = ID
	code(&ID) = 
	lis $3
	.word ____ (look up ID in symtable)
	add $3, $3, $29
	
	if expr = STAR factor:
	code(&*factor) = code(factor)
	\end{verbatim}
	
	\subsection*{New/Delete}
	\begin{itemize}
		\item runtime environment
		\item allocation module alloc.merl
		\item link to your assembled output along with print.merl
		\item MUST link it last
	\end{itemize}
	
\end{document}
