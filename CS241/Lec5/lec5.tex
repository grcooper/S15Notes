\documentclass[12pt]{article}

\setlength\parindent{0pt}
\newcommand{\myt}[1]{\textbf{\underline{#1}}}

\usepackage{mathtools}
\usepackage{amssymb}
\usepackage{listings}
\usepackage{mips}

\title{\vspace{-15ex}CS241 Lecture 5\vspace{-1ex}}
\date{May 20th, 2015}
\author{Graham Cooper}

\begin{document}
	\maketitle
	
	\section*{Assemblers}
	
	Assembly Code:\\
	add \$1, \$2, \$3\\
	jr \$31\\
	
	which goes into an assembler that outputs machine code.\\
	
	Any translation process involves 2 phases:\\
	\begin{enumerate}
		\item Analysis - Understanding what is meant by the source string (input)
		\item Synthesis - output the equivalent target-string
	\end{enumerate}
	
	assembly file $\rightarrow$ stream of characters\\
	- first step, generall,y is to group characters into meaningful tokens, eg. label, hex number, register number, .word directive, etc...\\
	- This is done for you\\
	
	My Job: Group tokens into instructions (if possible). (Analysis, trying to figure out what instructions are meant then outputing machine code)\\
	
	If tokens are not arranged into sensible instructions, output ERROR to stderr. Add more descriptive error messages if needed.\\
	
	Advice: There are many more wrong configurations that right ones.\\
	- Focus on finding all of the right ones - anything else is ERROR\\
	
	\subsection*{Problems}
	Biggest problem with writing assemblers:\\
	
	How do we assemble:
	\lstset{language=[mips]Assembler}
	\begin{lstlisting}
	beq $0, $1, abc
	...
	abc: add $3, $3, $3
	\end{lstlisting}
	The assembler does not know what ABC is yet, so it doesn't know the value of that label. We cant assemble the beq because we do not know what abc is yet.
	
	\myt{Standard solution:}\\
	Assemble in two passes.
	Pass1:\\
	-Group the tokens into instructions\\
	-Record the addresses of all labelled instructions (symbol table - (listof(label,address)))\\
	
	Notes:\\
	1) A line of assembly may have more than one label.\\
	eg:\\
	f:\\
	g:\\
	mult \$1, \$2\\
	2) The line after the last line can be labelled\\
	eg.\\
	jr \$31\\
	z:\\
	
	Pass2:\\
	- Translate each instruction into machine code
	- if an instruction refer
	s to a label, look up the associated address in the symbol table.\\
	
	Your assembler:\\
	- output assemble MIPS to stdout\\
	- output the symbol table to stderr\\
	
	Example:\\
	\begin{lstlisting}
	main: lis $2
	.word 13
	add $3, $0, $0
	top:
	add $3, $3, $2
	lis $1
	.word 1
	sub $2, $2, $1
	bne $2, $0, top
	jr $31
	beyond:
	\end{lstlisting}
	
	Pass1:\\
	- group tokens into instructions\\
	- build symbol table\\
	\begin{tabular}{c | c}
		name & address \\ \hline
		main & 0x00 \\
		top & 0x0c \\
		beyond & 0x24
	\end{tabular}
	
	Pass2: \\
	Translate each instruction\\
	eg. list $2 \rightarrow $ 0x00001014\\
	.word 13 $\rightarrow$ 0x0000000d\\
	...\\
	bne \$2, \$0, top\\
	- lookup top in symbol table which is 0x0c calculate $\frac{top-PC}{4} = -5$\\
	-get 0x1440fffb (-5 = 0xfffb)\\
	
	Note: To negate a 2's compliment number, flip the bits and add 1\\
	5 = 0000 000 000 0101\\
	-5 = 1111 1111 1111 1010 + 1\\
	= 1111 1111 1111 1011\\
	= fffb\\
	
	\myt{Bit level operations}\\
	To assemble bne \$2, \$0, top (where $\frac{top-PC}{4} = -5$)\\
	opcode = 000101 = 5\\
	first register \$2 = 2\\
	second register \$0 = 0\\
	offset = -5\\
	
	\begin{tabular}{|c | c | c | c |}
		6bits & 5 bits & 5 bits & 16 bits \\ \hline
	\end{tabular}
	To put 000101 in the first 6 bits we need to append 26 0's to it, ie left shift by 26 bits.\\
	C: 5 \textless \textless 26 (Racket: (arithmetic shift 5 -26))
	$\implies 000101\underset{26}{\underline{00...00}}$\\
	
	Move \$2, 21 bits to the left: 2 \textless \textless 21\\
	Move \$0, 16 bits to the left: 0 \textless \textless 16\\
	
	-5 = 0xfffffffb, we only want the last 16 bits in order to store our instruction properly.\\
	
	\begin{tabular}{c | c | c | c}
		a & b & a AND b & a OR b \\ \hline
		0 & 0 & 0 & 0 \\
		0 & 1 & 0 & 1 \\
		1 & 0 & 0 & 1 \\
		1 & 1 & 1 & 1 \\
	\end{tabular}
	
	extend these to bit-by-bit operations on full ints.\\
	
	$\frac{\underset{10010110}{11001010}}{10000010}$\\
	And with 1, get the other bit unchanged, AND with - get 0\\
	We can use AND to turn bits off, then we can use OR to turn bits on.\\
	
	So do a bitwise AND with 0x(0000)ffff
	c: -5 \& 0xffff\\
	Racket: (bitwise-and -5 \#xffff)\\
	
	Then bitwise-or the pieces together\\
	int x = $(5<<26)|(2<<21)|(0<<16)|(-5 \& 0xffff)$;\\
	= 339 804 155\\
	cout $<<$ x $<<$ endl; BAD DO NOT DOOOOO \\
	
	int x = 65;\\
	char c = 65;\\
	
	cout $<<$ x $<<$ c;\\
	Output: 65A\\
	
	int x = instruction;\\
	char c = x $>>$ 24;\\
	cout << c;\\
	c = x $>>$ 16;\\
	cout << c;\\
	c = x $>>$ 8;\\
	cout << c;\\
	c = x;\\
	cout << c;\\
	
	
	
	
	
\end{document}
