\documentclass[12pt]{article}

\setlength\parindent{0pt}
\newcommand{\myt}[1]{\textbf{\underline{#1}}}

\usepackage{mathtools}
\usepackage{amssymb}

\title{\vspace{-15ex}Lec18 CS241\vspace{-1ex}}
\date{July 8th, 2015}
\author{Graham Cooper}

\begin{document}
	\maketitle
	
	\section*{Grammar Rules}
	\subsection*{Expressions}
	Address of\\
	$\frac{E: int}{\&E : int *}$\\
	If E has type int, then \&E has type int *\\
	
	Dereference:\\
	$\frac{E: int *}{*E : int}$\\
	IF E has type int * then *E has type int\\
	
	Allocation:\\
	$\frac{E:\space int}{new int[E] : int *}$\\
	
	Mult:\\
	$\frac{E_1 :\space int\space E_2 : int}{E_1 * E_2 : int}$\\
	Similarily for /, \%\\
	
	Addition:\\
	$\frac{E_1: int \cap E_2:int}{E_1 + E_2 : int}$\\
	
	$\frac{E_1: int * \cap E_2:int}{E_1 + E_2 : int *}$\\
	
	$\frac{E_1: int \cap E_2:int *}{E_1 + E_2 : int *}$\\
	
	Substraction:\\
	$\frac{E_1: int \cap E_2:int}{E_1 + E_2 : int}$\\
	
	$\frac{E_1: int * \cap E_2:int}{E_1 + E_2 : int *}$\\
	
	$\frac{E_1: int * \cap E_2:int *}{E_1 + E_2 : int}$\\
	
	Functions:\\
	$\frac{<f,(\tau_1,\tau_2...\tau_n)> \in procs \space E_1: \tau_1 ... E_n: \tau_n}{F(E_1,...E_n): int}$\\
	
	\subsection*{Additional Type Constrains}
	Loops, if statements\\
	while(T) \{S\}\\
	if (T) \{$S_1$\} else \{$S_2$\}\\
	
	Test T hsould be boolean, not int or int*. But there is not boolean type in WLP4\\
	The grammar ensures that T will be a boolean tests (Expr comparison Expr), so as long as the comparison is well-typed, the if/while will be correct\\
	
	Any expression that has a type is well-typed.\\
	$\frac{E: \tau}{well-typed(E)}$\\
	
	Tests:\\
	$\frac{E_1: \tau \cap E_2 : \tau}{well-typed(E_1==E_2)}$\\
	similar with !=\\
	
	$\frac{E_1: \tau \cap E_2: \tau}{well-typed(E_1<E_2)}$\\
	Similar with $>,>=,<=$\\
	
	Assignment:\\
	$\frac{E_1: \tau \cap E_2: \tau}{well-typed(E_1 = E_2)}$\\
	
	Print:\\
	$\frac{E: int}{well-typed(println(E))}$\\
	
	Deallocation:\\
	$\frac{E: int*}{well-typed(delete[] E)}$\\
	
	IF:\\
	$\frac{well-typed(T) \cap well-typed(S_1) \cap well-typed(S_2)}{well-typed(if(T)(\{S_1\} else \{S_2\})}$\\
	
	Loops:\\
	$\frac{well-typed(T) \cap well-typed(S)}{well-typed(while(T)\{S\})}$\\
	
	Sequences:\\
	$\frac{}{well-typed(\epsilon)}$\\
	
	$\frac{well-typed(S_1) \cap well-typed(S_2)}{well-typed(S_1,S_2)}$\\
	
	Dcls:\\
	$\frac{}{well-typed(int id = NUM)}$\\
	$\frac{}{well-typed(int *id = NULL)}$\\
	
	Procedures:\\
	$\frac{well-typed(dcls) \cap well-typed(stmts) \cap E: int}{well-typed(INT, ID, (params)\{dcls, stmts, return, E; \})}$\\
	
	WAIN:\\
	$\frac{dcl_2=INT,ID \cap well-typed(dcls) \cap well-typed(stmts) \cap E: int}{well-typed(INT, WAIN(dcl, dcl_2)\{dcls, stmts, return, E;\})}$\\
	
	\section*{Code Generation}
	Source $\rightarrow$ Lexical Analysis $\overset{tokens}{\rightarrow}$ parsing $\rightarrow$ semantic analysys $\overset{parse tree and symbol table} {\rightarrow}$ Code Generation $\rightarrow$ assembly\\
	
	- By now, the source pgoram is guaranteed to be free of compile-time errors.\\
	- We must now output equivalent MIPS code\\
	
	There are $\infty$ many equivalent MIPS programs, which do we choose?\\
	\begin{itemize}
		\item correct (essential)
		\item easiest (for cs241)
		\item shortest
		\item fastest to run
		\item fastest to compile
	\end{itemize}
	
	\subsection{Example:}
	\begin{verbatim}
	int wain(int a, int b) { return a;}
	\end{verbatim}
	
	Convention:\\
	- parameters of wain are in \$1 and \$2\\
	- output will be in \$3\\
	
	MIPS:\\
	add \$3, \$1, \$0\\
	jr \$31\\
	
	\begin{verbatim}
	int wain (int a, int b) {return b;}
	\end{verbatim}
	MIPS:\\
	add \$3, \$2, \$0\\
	jr \$31\\
	
	Parse trees for these programs are the same.\\
	How do we tell one from the other?\\
	Determine which of the two declared ID's the returned ID matches\\
	
	How do we do that? Use the symbol table.\\
	Add a field to the symbol table to indicate where each symbol is stored\\
	\begin{tabular}{c | c | c}
		Name & TYpe & Location \\ \hline
		a & int & \$1 \\
		b & int & \$2 \\
	\end{tabular}
	
	Then when traversing for code gen, look up ID in the symbol table\\
	a $\rightarrow$ \$1\\
	b $\rightarrow$ \$2\\
	
	Local Declerations?\\
	- cant all be in regs, probably not enough\\
	For simplicity:\\
	\begin{itemize}
		\item put all local vars on the stack, including params of wain.
	\end{itemize}
	\begin{verbatim}
	int wain(int a, int b){return a;}
	\end{verbatim}
	
	\begin{verbatim}
	sw $1, -4 ($30)
	sw $2, -8($30)
	lis $4
	.word 8
	sub $30, $30, $4
	lw $3, 4($30)
	add $30, $30, $4
	jr $31
	\end{verbatim}
	
	
	
	
\end{document}
