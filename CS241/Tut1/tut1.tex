\documentclass[12pt]{article}

\setlength\parindent{0pt}
\newcommand{\myt}[1]{\textbf{\underline{#1}}}

\usepackage{mathtools}
\usepackage{amssymb}
\usepackage{listings}
\usepackage{mips}

\title{\vspace{-15ex}CS 241 Tutorial 1\vspace{-1ex}}
\date{May 15th, 2015}
\author{Graham Cooper}

\begin{document}
	\maketitle
	
	Fibonacci Numbers $f_n = f_{n-1} + f_{n-2}$\\
	where $f_0 = 1$ and $f_1 = 1$ and $n \geq 2$\\
	
	\myt{Topics}
	\begin{enumerate}
		\item MIPS Loops
		\item Printing and using the stack
		\item How to create and use procdedures
	\end{enumerate}
	
	\subsubsection*{1)}
	\$1 containts $n \geq 0$, find $f_n$ and place it in \$3
	
	\lstset{language=[mips]Assembler}
	\begin{lstlisting}
	; $3 = f_n
	; $4 = f_n-1
	add $3, $0, $0 ; begin $3 at 0
	lis $4
	.word 1 ; begin $4 at 1
	add $5, $0, $4 ; store 1 somewhere
	loop: beq $1, $0, ___ ; brance if our counter is 1
	add $4, $3, $4 ; put the next fib number in 4
	sub $1, $1, $5 ; n -= 1
	sub $3, $4, $3 ; 
	beq $0, $0, loop
	jr $31 ; return
	\end{lstlisting}
	
	\subsubsection*{2)}
	Convert problem 1 into a procedure name fib, apart from \$3, upon return, every register should have the ssame value upon return as it had when the procedure was called.\\
	
	\begin{lstlisting}
	fib: sw $4, -4($30) ; name our procedure
	sw $5, -8($30) ; store all of our used values in RAM
	sw $1, -12($30)
	lis $4
	.word 12
	sub $30, $30, $4 ; State how far we have used RAM
	add $3, $0, $0 ; begin $3 at 0
	lis $4
	.word 1 ; begin $4 at 1
	add $5, $0, $4 ; store 1 somewhere
	loop: beq $1, $0, ___ ; brance if our counter is 1
	add $4, $3, $4 ; put the next fib number in 4
	sub $1, $1, $5 ; n -= 1
	sub $3, $4, $3 ; 
	beq $0, $0, loop
	lis $4
	.word 12
	add $30, $30, $4 ; Restore the last used memory location
	lw $4, -4($30) ; restore all of our memory
	lw $5, -8($30)
	lw $1, -12($30)
	jr $31 ; return
	\end{lstlisting}
	
	\subsection*{Printing to Stdout and using the stack}
	- \$1 contains $n \geq 1$
	- using fib, print the first n fibonacci numbers in reverse
	- assume print procedure "print" uses \$1 as input
	
	\begin{lstlisting}
	lis $4
	.word fib
	lst $5
	.word print
	list $6
	.word 4
	sw $31, -4($30)
	sub $30, $30, $6
	add $7, $0, $0
	lis $8
	.word 1
	loop: beq $1, $7, endloop 
	add $7, $7, $8
	jalr $4, ; call fib
	sw $3, -4($30) ; store the result of fib in memory
	sub $30, $30, $6 ; add 4 to register 30
	beq $0, $0, loop
	endloop: 
	loop2: beq $1, $0, endloop2
	add $30, $30, $6
	lw $1 , -4($30)
	jalr $5
	sub $7, $7, $8
	beq $0, $0, loop2
	endloop2:
	add $30, $30, $6
	lw $31, -4($30)
	jr $31
	\end{lstlisting}
	
\end{document}
