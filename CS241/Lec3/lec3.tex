\documentclass[12pt]{article}

\setlength\parindent{0pt}
\newcommand{\myt}[1]{\textbf{\underline{#1}}}

\usepackage{mathtools}
\usepackage{amssymb}
\usepackage{listings}
\usepackage{mips}

\title{\vspace{-15ex}CS241 Lec 3\vspace{-1ex}}
\date{May 11th, 2015}
\author{Graham Cooper}

\begin{document}
	\maketitle
	
	Recall: Example 1: \$3 $\leftarrow $ \$5 + \$7\\
	
	Ex2: \$3 $\leftarrow$ 42 + 52\\
	
	lis \$d - load immediate \& skip\\
	
	\begin{tabular}{c | c | c | c}
		Location & Binary & Hex & Meaning \\ \hline
		00000000 & 0000 0000 0000 0000 0010 1000 0001 0100 & 0x00002814 & lis \$5 \\
		00000004 & 0000 0000 0000 0000 0000 0000 0010 1010 & 0x0000002a & .word 42 \\
		00000008 & 0000 0000 0000 0000 0011 1000 0001 0100 & 0x00003814 & lis \$7 \\
		0000000c & 0000 0000 0000 0000 0000 0000 0011 0100 & 0x00000034 & .word 52 \\
	\end{tabular}
	Then appaned code from Ex 1.
	
	\section*{Assembly Language}
	\begin{itemize}
		\item replace binary/hex encodings with easier to read shorthand
		\item less chance of error
		\item translation to binary can be automated (assembler)
		\item one line of assembly = 1 machine instruction (1 word)
	\end{itemize}
	
	\myt{EX2:}\\
	\lstset{language=[mips]Assembler}
	\begin{lstlisting}
		lis $5
		.word 42
		lis $7
		.word 52
		add $3, $5, 87
		jr $31
	\end{lstlisting}
	.word is not an instruction, it is a directive that teslls the assembler that the next word in the file should be literally 42//
	jr \$31 is return to loader\\
	
	\myt{Ex3}: Compute the absolute value of \$1, store in \$1, return\\
	\begin{itemize}
		\item some instructions modify PC
		\begin{itemize}
			\item jumps, eg jr
			\item branches
		\end{itemize}
		\item beq
		\begin{itemize}
			\item branch if 2 regs are equal
			\item increment PC by the given \# of words
			\item can branch backwards
		\end{itemize}
		\item also bne "Branch not equal"
		\item "set less than"
	\end{itemize}
	
	\begin{tabular}{c| c | c}
		address & code & literal \\ \hline
		00000000 & slt \$2, \$1, \$0 & compare \$1 \textless 0 \\
		00000004 & beq \$2, \$0, 1 & if the above is false, skip over \\
		00000008 & sub \$1, \$0, \$1 & \$1 $\leftarrow$ -\$1 \\
		0000000c & jr \$31 & exit program
	\end{tabular}
	\newline  \\
	
	\myt{Ex 4 (looping):} Sum 1 ... 13 store in \$3.\\
	
	\begin{tabular}{c | c | c}
		address & code & literal \\ \hline
		0 & lis \$2 & \$2 $\leftarrow$ 13 \\
		4 & .word 13 & \\
		8 & add \$3, \$0, \$0 & \$3 $\leftarrow$ 0 \\
		c & add \$3, \$2, \$3 & \$3 += \$2 \\
		10 & lis \$1 & \$1 $\leftarrow$ 1 \\
		14 & .word 1 & \\
		18 & sub \$2, \$2, \$1 & --\$2 \\
		1c & bne \$2, \$0, 5 & if \$2 $\neq$ \$0 back to line c \\
		20 & jr \$31 & exit program \\
	\end{tabular}
	
	\myt{RAM}\\
	\begin{itemize}
		\item lw "load-word" from RAM to regs
		\begin{itemize}
			\item lw \$a, i(\$b)
			\item \$a $\leftarrow$ MEM[\$b + i]
		\end{itemize}
		\item sw "store-word" - from regs to RAM
		\begin{itemize}
			\item sw \$a, i(\$b)
			\item MEM[\$b + i] $\leftarrow$ \$a
		\end{itemize}
	\end{itemize}
	
	\myt{Ex 5:} \$1 = address of an array, \$2 = length of the array
	Place element 5 (0-based) into \$3
	
	\myt{Easy Way:}\\
	\lstset{language=[mips]Assembler}
	\begin{lstlisting}
		lw $3, 20($1)
		jr $31
	\end{lstlisting}
	
	\myt{Hard Way:}\\
	Suppose \$5 contains the index of the item we want to fetch\\
	
	\lstset{language=[mips]Assembler}
	\begin{lstlisting}
		lis $4
		.word 4
		mult $5, $4
		mflo $5
		add $5, $1, $5
		lw $3, 0($5)
		jr $31
	\end{lstlisting}
	
	\subsubsection*{Mult Instruction Side Lesson}
	\begin{itemize}
		\item mult \$a, \$b
		\item product of 2 32-bit \#s is 64-bits (too big for a register)
		\item so two special registers, hi \& lo to store the result of a mult
		\item \$a, \$b $\equiv$ hi:lo $\leftarrow$ \$a $\times$ \$b
	\end{itemize}
	
	\subsubsection*{Revisit looping example}
	\lstset{language=[mips]Assembler}
	\begin{lstlisting}
		ls $2
		.word 13
		add $3, $0, $0
		add $3, $3, $2
		lis $1 (move above add)
		.word 1 (move above add)
		sub $2, $2, $1
		bne $2, $0, -5 (becomes -3)
		jr $31
	\end{lstlisting}
	
	Moving instructions into/out of branches, we must update branch offsets, and can be tricky\\
	
	Instead, the assembler allows labelled instructions\\
	label : instr\\
	
	eg: foo : add \$1, \$2, \$3\\
	Assembler associates the name 'foo' with the address of the command add \$1, \$2, \$3 in memory
	
	\lstset{language=[mips]Assembler}
	\begin{lstlisting}
		ls $2
		.word 13
		add $3, $0, $0
		top: add $3, $3, $2
		lis $1
		.word 1
		sub $2, $2, $1
		bne $2, $0, top
		jr $31
	\end{lstlisting}
	
	Assembler says top = 0x0c, but that will jump our program ahead?\\ 
	The assembler comutes the difference and jumps to the appropriate address. (Difference between PC and top in words)\\
	ie. $\frac{top-pc}{4} = \frac{0c - 20}{4} = \frac{12 - 32}{4} = -5$\\
	
\end{document}
