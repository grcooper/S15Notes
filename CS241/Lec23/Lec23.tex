\documentclass[12pt]{article}

\setlength\parindent{0pt}
\newcommand{\myt}[1]{\textbf{\underline{#1}}}

\usepackage{mathtools}
\usepackage{amssymb}

\title{\vspace{-15ex}CS 241 Lecture 23\vspace{-1ex}}
\date{July 27th, 2015}
\author{Graham Cooper}

\begin{document}
	\maketitle
	
	\section{Tail Recursion in WLP4}
	
	\begin{verbatim}
	int f(...) {
	-- if(...){
	-- -- if (..){	
	--	-- } else {}
	-- } 
	-- else {}
	return x;
	}
	\end{verbatim}
	
	Is the same as:\\
	
	\begin{verbatim}
	int f(...) {
	-- if(...){
	-- -- if (..){	
	--	-- } else {}
	-- return x;
	-- } 
	-- else {
	-- return x;
	--}
	}
	\end{verbatim}
	
	is the same as:\\
	\begin{verbatim}
	int f(...) {
	-- if(...){
	-- -- if (..){	
	-- -- return x; 
	-- -- } 
	-- -- else {
	-- -- return x;
	-- }
	-- return x;
	-- } 
	-- else {
	-- return x;
	--}
	}
	\end{verbatim}
	
	When return x follows an assignment to x, merge:\\
	x = f(...) $\rightarrow$ return f(...)\\
	return x;\\
	
	- may create some tail recursive calls\\
	
	Generalization:\\
	\begin{itemize}
		\item tail call optimization
		\item when a function's last action is any function call (recursive or not) can reuse the stack frame\\
	\end{itemize}
	
	\section*{Overloading}
	What would happen if we wanted to compile:\\
	
	\begin{verbatim}
	int f(int a){...}
	int f(int a, int *b){...}
	\end{verbatim}
	
	Get duplicated labels for f.\\
	How do we fix this?\\
	
	\subsection*{Name Mangling}
	Encode the types of params as part of the label\\
	
	Example naming convention:\\
	F + typeinfo + \_ + name\\
	ie.\\
	1. int f()\{...\}\\
	2. int f(int a)\{...\}\\
	3. int f(int a, int *b)\{...\}\\
	
	1. F\_f:\\
	2. Fi\_f:\\
	3. Fip\_f:\\
	
	
	\begin{itemize}
		\item C++ compilers wil ldo this because c++ has overloading
		\item there is no standard mangling convention
		\item all compilers are different
		\item makes it hard or  impossible to link code from different compilers
		\item this is by design b/c compilers differ in other aspects as well
	\end{itemize}
	
	C doesn't have overloading so there is no mangling\\
	- C and C++ code call each other routinely\\
	- How is this done?\\
	- Suppress mangling in c++\\
	Call C from C++\\
	- Extern "C" int f(int n); tells c++ f wot be mangled\\
	
	Call c++ from C - tell c++ not to mangle the function\\
	
	extern "C" int g(int x)\{...\} // dont mangle g\\
	
	and then obviously you cannot overload extern c functions\\
	
	\section*{Memory Management and the Heap}
	WLP4, C, C++\\
	\begin{itemize}
		\item explicit memory management
		\item user must free own data using free/delete
	\end{itemize}
	
	Java, Scheme\\
	\begin{itemize}
		\item implicit memory management
		\item garbage collection
	\end{itemize}
	
	\subsection*{How do new/delete or malloc/free work?}
	There are a variety of algorithms\\
	\subsubsection*{1)}
	List of free blocks:\\
	\begin{itemize}
		\item maintain a linked list of ptrs to blocks of free RAM
		\item Initially entire heap is free, list contains one entry
		\item Suppose heap is 1k
		\item suppose we allocate 16 bytes
		\begin{itemize}
			\item actually allocate 20 bytes + 1 int (4 bytes)
			\item return pointer to second word
			\item store size just before the returned pointer\
			\item free list ctonains the rest of the loop
		\end{itemize}
	\end{itemize}
	
	\myt{Note}: Repeated allocation and deallocation creates "holes" in the heap\\
	\myt{EG:}\\
	\begin{verbatim}
	alloc 20 {xx 20 xx,... (140)... ...}
	alloc 40 {xx 20 xx, xx 40 xx,...(100) ... ...}
	alloc 20 {...(20)..., xx 40 xx, ... (100)...}
	alloc 5 {xx 5 xx, ... (15). .., xx 40 xx, ... (100)...}
	etc.
	\end{verbatim}
	
	We get holes like the 15 block hole on the last line, this causes:\\
	\myt{Code fragmentation} - means even if n bytes are free, we may not be able to allocate n bytes\\
	
	To reduce fragmentation:\\
	- don't always pick the first block of RAM big enought to satisfy the request\\
	
	\subsubsection*{2) Binary Buddy System}
	Assume: size of heap is a power of 2.\\
	Example: heap is 4k (4096 byes) = 1024 words.\\
	
	Suppose
	\begin{itemize}
		\item program requests 20 words
		\item we need 1 word for bookeeping
	\end{itemize}
	
	Memory allocated in blocks of size 2$^k$
	\begin{itemize}
		\item so allocate 23 words
		\item 1024 is too big so split into two heaps (buddies)
		\item allocate from one of the 512-word buddies
		\item still too big, split one of them again
		\item still to big, split the 256 word buddies
		\item continue to split until one of the blocks is the size you need for your 21 words
		\item we return a pointer to a block of 32 words, with the first one being for bookeeping
	\end{itemize}
	Now we request 63 words, so we allocate 64
	\begin{itemize}
		\item heap contains a 64-word block
		\item we allocate that 64-word block
	\end{itemize}
	
	Now request 50 words, llocate for 51\\
	\begin{itemize}
		\item split one of the 128 blocks into 64
		\item allocate one of the new 64 blocks
	\end{itemize}
	
	First 64 word block is released\\
	\begin{itemize}
		\item simply just free the memory
	\end{itemize}
	
	Free the first 32-word block\\
	\begin{itemize}
		\item we free the 32 block and see that it's buddy is also free
		\item since they are both free, merge them into a 64 word block
		\item now the 64 block and its buddy is free, so merge into 128 block
	\end{itemize}
	
	Finally free the last 64-word block
	\begin{itemize}
		\item merge the 64 block that was just freed and its buddy
		\item we can now merge the rest of the blocks since everything is free into the 1024 block again
	\end{itemize}
	
	\subsection*{Deallocation info}
	alloc.asm
	\begin{itemize}
		\item assignms each block a code
		\item entire heap has a code of 1
		\item the 512-word buddies have codes of 10, and 11
		\item 256 word buddies have codes of 100, 101, 110, 111
		\item these are all encoded like a trie
		\item to find your buddy's code : flip te last bit
		\item mergin buddies - drop the last big ($>>$1)
	\end{itemize}
	
	\section*{Implicity Memory Management: Garbage Collection}
	System reclaims memory once client can no longer access it\\
	\subsection*{1) Mark and Sweep}
	\begin{itemize}
		\item scan the entire stack looking for pointers
		\item for each pointer found mark the heap block its pointing to
		\item then if the heap block contains pointers, follow those as well and mark etc...
	\end{itemize}
	\begin{itemize}
		\item scan the heap
		\item reclaim any blocks not marked
		\item clear all marks
	\end{itemize}
	
	\subsection*{2) Reference Counting}
	\begin{itemize}
		\item for each heap block, keep track of the number of pointers that point to it (reference count)
		\item means you must eatch every pointer and update ref counts (decrement old, increment new) each time a pointer is reassigned
		\item if a block's ref count reaches 0, reclaim
	\end{itemize}
	There is a problem when we have circular references, each is a pointer on the heap, pointing at each other, their reference count is 0, but there is nothing on the stack pointing at them\\
	
	\subsection*{3) Copying GC}
	Heap is in two halves, FROM and TO\\
	\begin{itemize}
		\item allocate only from FROM
		\item when FROM fills up, all reachable data is copied from FROM to TO
		\item Built in Compaction
		\item guaranteed that after the swap, the compactor will make it such that all reachable data occupies memory
		\item this causes there to be no fragmentation yay!
		\item BUT we can only use half of the heap at a time
	\end{itemize}
	
\end{document}
