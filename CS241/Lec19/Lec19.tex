\documentclass[12pt]{article}

\setlength\parindent{0pt}
\newcommand{\myt}[1]{\textbf{\underline{#1}}}

\usepackage{mathtools}
\usepackage{amssymb}

\title{\vspace{-15ex}CS241 Lecture 19\vspace{-1ex}}
\date{July 13th, 2015}
\author{Graham Cooper}

\begin{document}
	\maketitle
	Recal:\\
	\begin{itemize}
		\item put all local vars on the stack, including params of wain.
	\end{itemize}
	\begin{verbatim}
	int wain(int a, int b){return a;}
	\end{verbatim}
	
	\begin{verbatim}
	sw $1, -4 ($30)
	sw $2, -8($30)
	lis $4
	.word 8
	sub $30, $30, $4
	lw $3, 4($30)
	add $30, $30, $4
	jr $31
	\end{verbatim}
	
	\begin{tabular}{c | c | c}
		name & type & offset from \$30 \\ \hline
		a & int & 4 \\
		b & int & 0\\
	\end{tabular}
	
	\myt{Problem:} Can't know the offsets until all decls have been processed because \$30 changes with each new decleration\\
	
	\begin{verbatim}
	int wain(int a, int b){
	int c = 0;
	return a;
	}
	\end{verbatim}
	
	\begin{tabular}{c | c | c}
		name & type & offset from \$30 \\ \hline
		a & int & 8 \\
		b & int & 4 \\
		c & int & 0 \\
	\end{tabular}
	
	Introduce two conventions:\\
	\begin{itemize}
		\item \$4 always contains 4
		\item \$29 points to the bottom of the stack frame
		\item if offsets are calculated with relation to \$29 they will be constant
	\end{itemize}
	
	\begin{verbatim}
	int waint(int a, int b){
	int c = 0;
	return a;
	}
	\end{verbatim}
	
	\begin{verbatim}
	lis $4
	.word 4
	sub $29, $30, $4
	sw $1, -4($30)
	sw $2, -8($30)
	sw $0, -12($30)
	lis $5
	.word 12
	sub $30, $30, $12
	lw $3, 0($29)
	add $30, $30, $5
	jr $31
	\end{verbatim}
	
	\begin{tabular}{c | c | c }
		name & typ & offset from \$29 \\ \hline
		a & int & 0 \\
		b & int & -4 \\
		c & int & -8 \\
	\end{tabular}
	
	\$29 is called the frame pointer\\
	What about more complicated programs?\\
	
	\begin{verbatim}
	int wain(int a, int b){
		return a + b;
	}
	\end{verbatim}
	In general: For every grammar rule A $\rightarrow \gamma$ build code (A) from code ($\gamma$)\\
	
	Extend Previous Convention:\\
	Use \$3 as "output" for every expression\\
	
	problem:\\
	a + b:\\
	\$3 $\leftarrow$ eval(a)\\
	\$3 $\leftarrow$ eval(b)\\
	\$3 $\leftarrow$ \$3 + \$3\\
	value of a is lost!!\\
	
	Need a place to store pending computation. Could use a register:\\
	\begin{verbatim}
	code(a+b) = 
	code(a) ($3 = a)
	add $5, $3, $0 ($5 = a)
	code(b) ($3 = b)
	add $3, $5, \$3
	\end{verbatim}
	
	What about a + (b + c)\\
	\begin{verbatim}
	code(a) ($3 = a)
	add $5, $3, $0 ($5 = a)
	code(b) ($3 = b)
	add $6, $3, $0 ($6 = b)
	code(c) ($3 = c)
	add $3, $6, $3 ($3 = b + c)
	add $3, $5, $3 ($3 = a + (b + c))
	\end{verbatim}
	
	Similarily, a + (b + (c + d)) woudl need 3 edtra regs, we would eventualyl run out of registers.\\
	More general solution: use the stack\\
	\begin{verbatim}
	code(a) ($3 = a)
	push($3)
	code(b) ($3 = b)
	push($3)
	code(c) ($3 = c)
	push($3)
	code(d) ($3 = d)
	pop($5) ($5 = c)
	add $5, $5, $3 ($3 = c +d)
	pop($5) ($5 = b)
	add $3, $5, $3 ($3 = b + (c + d))
	pop($5) ($5 = a)
	add $3, $5, $3 ($3 = a + (b + (c + d)))
	\end{verbatim}
	
	In general:\\
	expr1 -$>$expr2 + term\\
	code(expr1) = code(expr2)\\
	+push(\$3)\\
	+code(term)\\
	+pop(\$5)\\
	+add \$3, \$5, \$3\\
	
	Singleton rules: easy\\
	expr -$>$term: code(expr) = code(term)\\
	
	\myt{Print:} println(expr) prints expr, followed by newline\\
	You did this already! A2P6, A2P7a\\
	
	\myt{Runtime Environment:} set of procedures supplied by the compiler (or the OS) to assist programs in their exectuion\\
	msvcrt.dll\\
	- procedure print will be part of the runtime evironment.\\
	- We will provide print.merl\\
	
	./wlp4gen $<$ prog.wlp4i $>$ prog.asm\\
	java cs241.linkasm $<$ prog.asm $>$ prog.merl\\
	linker prog.merl print.merl $>$ exec.merl\\
	java mips.\{twoints or array\}exec.merl\\
	
	We can assume print is available it takes input int \$1.\\
	
	code(println(expr)) = code(expr) (\$3 = expr)\\
	+ add \$1, \$3, \$0\\
	+ call print\\
	
	- remember to save and restore \$31\\
	
	\section*{Assignment:}
	stmt $\rightarrow$ lvalue = expr\\
	for now: assume lvalue is ID (no ptrs yet)\\
	
	\begin{verbatim}
	stmt -> lvalue = expr
	cod(stmt) = code(expr) ($3 <- expr)
	sw $3, _($29)
	\end{verbatim}
	
	The blank spot above we look up ID's offset in symbol table\\
	
	What's left - if and while\\
	- need boolean testing\\
	Suggest:\\
	Store the number 1 in register 11\\
	May want to store print in register 10\\
	Code so far:\\
	\begin{verbatim}
	.import print
	list $4
	.word 4
	list $10
	.word print
	lis $11
	,word 1
	sub $29, $30, $4
	(put all params, locals on stack)
	MY CODE WOOO
	add $30, $29, $4
	jr $31
	\end{verbatim}
	
	Stuff before my code is the prologue, stuff after my code is the epilogue\\
	
	\section*{boolean testing}
	\begin{verbatim}
	test -> expr1 < expr2
	
	code(test) = code(expr1) ($3 <- expr1)
	add $6, $3, $0 ($6, <- expr1)
	code(expr2) ($3 <- expr2)
	slt $3, $6, $3 ($3 <- expr1 < expr2)
	\end{verbatim}
	
	\begin{verbatim}
	test -> expr1 > expr2
	treat as expr1 as above\\
	\end{verbatim}
	
\end{document}
